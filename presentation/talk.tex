\documentclass{beamer}
\mode<presentation>

%Removes some warnings
\let\Tiny=\tiny

\usepackage{multicol}
\usepackage{caption}
\usepackage{subcaption}
\usepackage{textpos}
\usepackage{graphicx}

\usepackage{tikz}
\usetikzlibrary{arrows,shapes}

\usetheme{Darmstadt}

%% http://www.math.umbc.edu/~rouben/beamer/

%Additional theme/color changes
%\definecolor{umn_maroon}{rgb}{0.48,0,0.01}
\definecolor{umn_maroon}{rgb}{0.39,0,0.08}
\definecolor{umn_maroon_dark}{rgb}{1,0.2,0.13}
\definecolor{umn_gold}{rgb}{1,0.8,0.2}

%Title Page
\setbeamercolor{titlelike}{bg=umn_maroon,fg=umn_gold}
\setbeamercolor{section in toc}{fg=umn_maroon}

%Frame headers
\setbeamercolor{section in head/foot}{fg=umn_gold}
\setbeamercolor{subsection in head/foot}{bg=umn_gold}
\setbeamercolor{subsection in head/foot}{fg=umn_gold}
\setbeamercolor{navigation symbols}{fg=umn_maroon,bg=umn_maroon_dark}
\setbeamercolor{frametitle}{bg=umn_maroon}
\setbeamercolor{frametitle}{fg=umn_gold}

%Itemize/Enumerates
\setbeamercolor{item}{fg=umn_maroon}

%
\setbeamercolor{caption}{{fg=black}}
\setbeamercolor{caption name}{{fg=black}}

\setbeamerfont{caption}{size=\tiny}

\setbeamertemplate{navigation symbols}{}

\addtobeamertemplate{frametitle}{}
{%
  \begin{textblock*}{100mm}(0.76\textwidth,-0.6cm)
    \includegraphics[height=0.35cm,width=3.45cm]{diagrams/template_graphics/goldWMlong.png}
  \end{textblock*}
}

% For every picture that defines or uses external nodes, you'll have to
% apply the 'remember picture' style. To avoid some typing, we'll apply
% the style to all pictures.
\tikzstyle{every picture}+=[remember picture]

% By default all math in TikZ nodes are set in inline mode. Change this to
% displaystyle so that we don't get small fractions.
\everymath{\displaystyle}


% Remove "Fig." from figure captions
\renewcommand{\figurename}{}
\renewcommand{\subfigurename}{}


\title[]{Interactive Image Segmentation Using Texture}
\author{Robert F.K. Martin}
\institute{Department of Computer Science \\ University of Minnesota}
\date{June 11, 2013}

\begin{document}

%%v-------------------------------------------------------------------v%
%
%
\begin{frame}
  \titlepage
\end{frame}
%%
%%^-------------------------------------------------------------------^

%%v-------------------------------------------------------------------v%
%
%
\begin{frame}{Outline}
  \tableofcontents[section,subsection]
\end{frame}
%%
%%^-------------------------------------------------------------------^
\section{Motivation}
\subsection{}
%%v-------------------------------------------------------------------v%
%
%%
\begin{frame}{Motivation}
   \begin{figure}
      \includegraphics[scale=0.5]{bai.png}
   \end{figure}
  {\scriptsize X.~Bai and G.~Sapiro, ``A geodesic framework for fast interactive image and
  video segmentation and matting,'' in \emph{Proc. of the 11th IEEE
  International Conference on Computer Vision}, October 2007, pp. 1--8.}
  \begin{itemize}
    \item User scribbles denote foreground/background
    \item Automatically separates objects of interest from everything else(background/noise)
  \end{itemize}
\end{frame}
%%
\begin{frame}{Motivation}
\begin{columns}[c]
\column{3.0in}
\begin{itemize}
    \item Segmentation is handy but limited if it relies on color
     \item Many sensors have only one channel
     \begin{itemize}
      \item Medical images(MRI, X-ray)
      \item Satellite images
      \item Low-cost cameras
     \end{itemize}
  \end{itemize}
\column{36mm}
\begin{center}
  \includegraphics[width=\textwidth,height=0.5\textheight,keepaspectratio]{teeth.png}\\
  \includegraphics[scale=0.03]{san_ysidro_nm.jpg}
\end{center}
\end{columns}
\end{frame}
%%
%%^-------------------------------------------------------------------^
\section{Background}
\subsection{}
%%v-------------------------------------------------------------------v%
%
\begin{frame}{Background}
Interactive Segmentation
   \begin{itemize}
     \item Interactive Graph Cuts -- Boykov, Jolly; 2001
     \item Iterative Figure-Ground Discrimination -- Zhao, Davis; 2004
     \item Random Walks-- Grady; 2006
     \item Grabcut
     \item Bai, Sapiro
   \end{itemize}
\end{frame}
%
\begin{frame}{Background}
   Texture Segmentation
   \begin{itemize}
     \item Jain, Farrokhnia
     \item Carevic
   \end{itemize}
\end{frame}
%
%%
%%^-------------------------------------------------------------------^
\section{Methods}
\subsection{}
\begin{frame}{Methods}
General order of operation
      \includegraphics[width=34mm]{../knut_color_input.jpg}
      \includegraphics[width=40mm]{../knut_divf.png}
      \includegraphics[width=34mm]{../knut_color_final.jpg}
   \begin{itemize}
   \item Create marks on the image to distinguish foreground/background
   \item Gather underlying image statistics from those samples(texture,kernel density estimation)
   \item Separate foreground from background(geodesic distance)
   \end{itemize}
\end{frame}

%%
%%^-------------------------------------------------------------------^
\subsection{Geodesic Distance}
%%v-------------------------------------------------------------------v%
%
\begin{frame}{Geodesic Distance}
Geodesics are shortest lines that connect two points on a curved surface. The likelihood is
\begin{equation}L_{x}=\frac{P_{x}}{P_{f} + P_{b}}\end{equation}
which is the probability that a pixel is a member of the foreground of background. The gradiant of the likelihood then defines the surface over which we attempt to move from point A(user scribbles) to point B(all other points).
\end{frame}
%
\begin{frame}{Geodesic Distance}
\begin{itemize}
\item Geodesic distances were calculated using Fast Forward Marching.
\item Can be calculated in a linear time fashion, visiting every pixel only once each for foreground and background
\item Each pixel's membership is determined by the lesser of the two geodesic distances
\end{itemize}
\end{frame}
%
%%^-------------------------------------------------------------------^
\subsection{Texture}
%%v-------------------------------------------------------------------v%
%
\begin{frame}{Texture}
Texture is a difficult concept to define.
   \begin{quote}
      The visual or tactile surface characteristics and appearance of something.\\
      \hfill{{\scriptsize \emph{Merriam-Webster Dictionary}}}
   \end{quote}
   \begin{quote}
      Image texture is a set of metrics calculated in image processing designed to quantify the perceived texture of an image.\\
      \hfill{{\scriptsize Linda G. Shapiro and George C. Stockman, \emph{Computer Vision}, Upper Saddle River: Prentice�Hall, 2001}}
   \end{quote}
\end{frame}
%
\begin{frame}{Texture}
Some easily computed and commonly used metrics
   \begin{itemize}
      \item edges or corners(strength,edginess)
      \item correlation
      \item image intensity
      \item Fourier response
   \end{itemize}
\end{frame}
%
\begin{frame}{Texture}
Gabor Filters
\begin{itemize}
   \item 2D filter generated by Gaussian kernel modulated by sinusoid
   \item biologically inspired - similar responses to some visual cortex cells
   \item have been used for texture analysis and segmentation
\end{itemize}
\end{frame}
%
\begin{frame}{Texture}
Gabor Filter Equation
$g(x,y,\lambda,\theta,\psi,\sigma,\gamma) = \exp\left(-\frac{x^{\prime 2}+\gamma^2 y^{\prime 2}}{2 \sigma^2}\right)\cos \left(2 \pi\frac{x^\prime}{\lambda} + \psi\right)$
\begin{itemize}
   \item $\lambda$: wavelength of the sinusoid
   \item $\theta$: orientation from the normal
   \item $\sigma$: Gaussian window
   \item $\psi$: phase offset
   \item $\gamma$: ellipticity
\end{itemize}
\end{frame}
%
\begin{frame}{Texture}
\begin{itemize}
\item Filters were created by varying orientation($\theta$), frequency($\lambda$), and envelope size($\sigma$).\\
\item A $\frac{\pi}{2}$ offset was chosen to give oriented edge detection.\\
\item Eight values each were selected for orientation, frequency, and Gaussian window, giving a bank of 512 different filters.\\
\item Each filter was normalized to ensure similar responses.\\
\item At each pixel, the filter values for the maximum response were substituted in for the image's {\it RGB} values.
\item These pseudocolor images were then used as inputs to segmentation
\end{itemize}
\end{frame}
%
\begin{frame}{Texture}
\begin{center}
\LARGE{Selected Gabor patches}
   \begin{figure}
      \includegraphics[scale=0.5]{array.png}
   \end{figure}
\end{center}
\end{frame}

\subsection{Kernel Density Estimation}
%%v-------------------------------------------------------------------v%
%
\begin{frame}{Kernel Density Estimation}
\begin{equation}\label{kde}p(x)=\frac{1}{N}\sum_{i=1}^{N}{K_{\sigma}(x-x_i)}\end{equation}
\begin{itemize}
   \item $K_\sigma$ -- the kernel, or windowing function
   \item $\sigma$ -- bandwidth
   \item $N$ -- number of samples
   \end{itemize}
\end{frame}
%
\begin{frame}{Kernel Density Estimation}
Kernels must be:
\begin{itemize}
   \item of unit area
   \item symmetric about 0
   \end{itemize}
These guarantee that a kernel will converge to the true PDF given enough samples. Typically a gaussian is used although other kernels are available(uniform, triangular, Epanechnikov).
\end{frame}
%
\begin{frame}{Kernel Density Estimation}
\begin{center}
      \includegraphics[height=40mm]{../texture3_color_input.jpg} \\
      \includegraphics[height=35mm]{sigma.png}
      \includegraphics[height=35mm]{theta.png}
      \includegraphics[height=35mm]{lambda.png}
   \end{center}
\end{frame}
%
\begin{frame}{Kernel Density Estimation}
   \begin{center}
      \includegraphics[width=120mm]{histogram.png}
   \end{center}
\end{frame}
%
%%
%%^-------------------------------------------------------------------^
\section{Results}
\subsection{}
%%v-------------------------------------------------------------------v%
%
\begin{frame}{Results}
Start with a simple case and verify color segmentation works
   \begin{center}
      \includegraphics[height=36mm]{../knut_color_input.jpg}
      \hfill \hfill
      \includegraphics[height=36mm]{../knut_color_final.jpg}
   \end{center}
\end{frame}

\begin{frame}{Results}
\LARGE{Input Images}
   \begin{figure}
      \includegraphics[height=36mm]{../../data/texture1.jpg}
      \includegraphics[height=36mm]{../../data/texture2.jpg}
      \includegraphics[height=36mm]{../../data/texture3.jpg}
   \end{figure}
\end{frame}
%
\begin{frame}{Results}
\LARGE{Input Images -- Marked Up}
   \begin{figure}
      \includegraphics[height=36mm]{../texture1_color_input.jpg}
      \includegraphics[height=36mm]{../texture2_color_input.jpg}
      \includegraphics[height=36mm]{../texture3_color_input.jpg}
   \end{figure}
\end{frame}
%
\begin{frame}{Results}
\LARGE{Pseudocolor Texture}
   \begin{figure}
      \includegraphics[height=36mm]{../texture1_gabor_pseudocolor.jpg}
      \includegraphics[height=36mm]{../texture2_gabor_pseudocolor.jpg}
      \includegraphics[height=36mm]{../texture3_gabor_pseudocolor.jpg}
   \end{figure}
\end{frame}
%
\begin{frame}{Results}
\LARGE{Color Segmentation Results}
   \begin{figure}
      \includegraphics[height=36mm]{../texture1_color_final.jpg}
      \includegraphics[height=36mm]{../texture2_color_final.jpg}
      \includegraphics[height=36mm]{../texture3_color_final.jpg}
   \end{figure}
\end{frame}
%
\begin{frame}{Results}
\LARGE{Texture Segmentation Results}
   \begin{figure}
      \includegraphics[height=36mm]{../texture1_gabor_final.jpg}
      \includegraphics[height=36mm]{../texture2_gabor_final.jpg}
      \includegraphics[height=36mm]{../texture3_gabor_final.jpg}
   \end{figure}
\end{frame}%
%
\begin{frame}{Results}
\LARGE{Input}
   \begin{center}
      \includegraphics[scale=0.5]{../stripes_color_input.jpg}
   \end{center}
\end{frame}%
%
\begin{frame}{Results}
\LARGE{Color Segmentation Results}
   \begin{center}
      \includegraphics[scale=0.5]{stripes_color_final.jpg}
   \end{center}
\end{frame}%
%
\begin{frame}{Results}
\LARGE{Texture Pseudocolor}
   \begin{center}
      \includegraphics[scale=0.5]{../stripes_gabor_pseudocolor.jpg}
   \end{center}
\end{frame}%
%
\begin{frame}{Results}
\LARGE{Texture Segmentation Results}
   \begin{center}
      \includegraphics[scale=0.5]{../stripes_gabor_final.jpg}
   \end{center}
\end{frame}%
%%
%%^-------------------------------------------------------------------^
\section{Conclusion}
\subsection{}
%%v-------------------------------------------------------------------v%
%
\begin{frame}{Conclusion}
\begin{itemize}
\item In certain cases, segmentation using texture provides much better results than using color
\item Using texture as a color proxy results in a powerful new way to handle segmentation of monochrome images
\end{itemize}
\end{frame}
%
\begin{frame}{Thank You}
   \begin{center}
      \huge{Questions?}
   \end{center}
\end{frame}
%
%%
%%^-------------------------------------------------------------------^
\end{document}